% Options for packages loaded elsewhere
\PassOptionsToPackage{unicode}{hyperref}
\PassOptionsToPackage{hyphens}{url}
%
\documentclass[
  ignorenonframetext,
]{beamer}
\usepackage{pgfpages}
\setbeamertemplate{caption}[numbered]
\setbeamertemplate{caption label separator}{: }
\setbeamercolor{caption name}{fg=normal text.fg}
\beamertemplatenavigationsymbolsempty
% Prevent slide breaks in the middle of a paragraph
\widowpenalties 1 10000
\raggedbottom
\setbeamertemplate{part page}{
  \centering
  \begin{beamercolorbox}[sep=16pt,center]{part title}
    \usebeamerfont{part title}\insertpart\par
  \end{beamercolorbox}
}
\setbeamertemplate{section page}{
  \centering
  \begin{beamercolorbox}[sep=12pt,center]{part title}
    \usebeamerfont{section title}\insertsection\par
  \end{beamercolorbox}
}
\setbeamertemplate{subsection page}{
  \centering
  \begin{beamercolorbox}[sep=8pt,center]{part title}
    \usebeamerfont{subsection title}\insertsubsection\par
  \end{beamercolorbox}
}
\AtBeginPart{
  \frame{\partpage}
}
\AtBeginSection{
  \ifbibliography
  \else
    \frame{\sectionpage}
  \fi
}
\AtBeginSubsection{
  \frame{\subsectionpage}
}
\usepackage{lmodern}
\usepackage{amsmath}
\usepackage{ifxetex,ifluatex}
\ifnum 0\ifxetex 1\fi\ifluatex 1\fi=0 % if pdftex
  \usepackage[T1]{fontenc}
  \usepackage[utf8]{inputenc}
  \usepackage{textcomp} % provide euro and other symbols
  \usepackage{amssymb}
\else % if luatex or xetex
  \usepackage{unicode-math}
  \defaultfontfeatures{Scale=MatchLowercase}
  \defaultfontfeatures[\rmfamily]{Ligatures=TeX,Scale=1}
\fi
% Use upquote if available, for straight quotes in verbatim environments
\IfFileExists{upquote.sty}{\usepackage{upquote}}{}
\IfFileExists{microtype.sty}{% use microtype if available
  \usepackage[]{microtype}
  \UseMicrotypeSet[protrusion]{basicmath} % disable protrusion for tt fonts
}{}
\makeatletter
\@ifundefined{KOMAClassName}{% if non-KOMA class
  \IfFileExists{parskip.sty}{%
    \usepackage{parskip}
  }{% else
    \setlength{\parindent}{0pt}
    \setlength{\parskip}{6pt plus 2pt minus 1pt}}
}{% if KOMA class
  \KOMAoptions{parskip=half}}
\makeatother
\usepackage{xcolor}
\IfFileExists{xurl.sty}{\usepackage{xurl}}{} % add URL line breaks if available
\IfFileExists{bookmark.sty}{\usepackage{bookmark}}{\usepackage{hyperref}}
\hypersetup{
  pdftitle={Workshop 3},
  pdfauthor={Bolun},
  hidelinks,
  pdfcreator={LaTeX via pandoc}}
\urlstyle{same} % disable monospaced font for URLs
\newif\ifbibliography
\setlength{\emergencystretch}{3em} % prevent overfull lines
\providecommand{\tightlist}{%
  \setlength{\itemsep}{0pt}\setlength{\parskip}{0pt}}
\setcounter{secnumdepth}{-\maxdimen} % remove section numbering
\ifluatex
  \usepackage{selnolig}  % disable illegal ligatures
\fi

\title{Workshop 3}
\author{Bolun}
\date{01/27/2021}

\begin{document}
\frame{\titlepage}

\begin{frame}{Review}
\protect\hypertarget{review}{}
\begin{itemize}
\item
  Let me preach a little bit (again)
\item
  Population and sample
\item
  Statistic inference: from sample to population
\item
  Normal distribution
\item
  Before we enter into the discussion, let's review the codes in small
  groups (and thank Zahra!).
\end{itemize}
\end{frame}

\begin{frame}{Importance of the randomness}
\protect\hypertarget{importance-of-the-randomness}{}
Group Discussion:

\begin{itemize}
\tightlist
\item
  Why randomness is important for inference?
\item
  What kind of normative principle lies behind it?
\item
  What if a sample is conentrate among certain group of people?
\end{itemize}

BTW: Beware! When people talk about inference/prediction, they might
mean a lot of different things.
\end{frame}

\begin{frame}{Sampling Distribution 1}
\protect\hypertarget{sampling-distribution-1}{}
What is sampling distribution: The probability distribution of a given
statistics (mean, sd etc.).

Here using sampling distribution of the mean as an example.

Imagine that you have a given population of 1000 people, each have a
given value of asset. You randomly selected 100 from them, and calculate
the mean of this sample. Then, you repeat this process for almost
infinite time, the probability distribution you come up with in the end
is a sampling distribution of the mean.
\end{frame}

\begin{frame}{Sampling Distribution 2}
\protect\hypertarget{sampling-distribution-2}{}
Let's simulating it in codes.
\end{frame}

\begin{frame}{Standard Error 1}
\protect\hypertarget{standard-error-1}{}
Standard error is the standard deviation of the sampling distribution.

Mathematically you can derive that

\[\sigma_{\bar{x}} =  \frac{\sigma}{\sqrt{n}}\]

Where \(\sigma\) is the standard deviation of the population, and \(n\)
is the size of the sample.

In a lot of situation, if we know the kind of distribution the
population belongs, we can infer the sampling distribution
mathematically. In the case of mean estimate, the sampling distribution
is a normal distribution.
\end{frame}

\begin{frame}{Standard Error 2}
\protect\hypertarget{standard-error-2}{}
But usually we do not know the standard deviation of the population as
we did in the last simulation. Thus, we usually estimate the standard
error using the standard deviation of the sample.

\[\sigma_{\bar{x}} \approx \frac{s}{\sqrt{n}}\]

or a old fashioned way

\[\sigma_{\bar{x}} \approx \frac{s}{\sqrt{n-1}}\]
\end{frame}

\begin{frame}{Normal Distribution and Student Distribution}
\protect\hypertarget{normal-distribution-and-student-distribution}{}
Review: normal distribution and student distribution.

Two different situations:

\begin{enumerate}
[1)]
\item
  When the sample size is large enough (n \textgreater{} 30), the
  sampling distribution is also a normal distribution.
\item
  When our sample size is small, the standard error estimate is not
  accurate enough. In this situation, we do not use the normal
  distribution for the purpose of statistic inference. Instead, we use
  student distribution.
\end{enumerate}
\end{frame}

\begin{frame}{Group Activities}
\protect\hypertarget{group-activities}{}
Group Activity: Changing the codes and modify the size of the sample to
a small number, see how the distribution changed.
\end{frame}

\begin{frame}{Mean point estimate and confidential intervals 1}
\protect\hypertarget{mean-point-estimate-and-confidential-intervals-1}{}
Group Activity:

\begin{itemize}
\tightlist
\item
  Take a large n size sample from the population we generated
  previously, using simulation to generate a distribution of (sample
  mean - population mean).
\item
  Use normal distribution to calculate the 95\% confidential intervals
  for the point estimate.
\item
  Compare the results
\end{itemize}
\end{frame}

\begin{frame}{Mean point estimate and confidential intervals 2}
\protect\hypertarget{mean-point-estimate-and-confidential-intervals-2}{}
Group Activity:

\begin{itemize}
\tightlist
\item
  Take a small n size sample from the population we generated
  previously, using simulation to generate a distribution of (sample
  mean - population mean).
\item
  Use student distribution to calculate the 95\% confidential intervals
  for the point estimate.
\item
  Compare to previous result, what do you find?
\end{itemize}
\end{frame}

\begin{frame}{Ending: from estimation to hypothesis testing}
\protect\hypertarget{ending-from-estimation-to-hypothesis-testing}{}
\begin{itemize}
\tightlist
\item
  Similar logic
\item
  Clarification about confidential intervals:

  \begin{itemize}
  \tightlist
  \item
    ``A 95\% confidence level does not mean that for a given realized
    interval there is a 95\% probability that the population parameter
    lies within the interval (i.e., a 95\% probability that the interval
    covers the population parameter).''
  \item
    It's about the difference between the estimate and the true
    parameter.
  \end{itemize}
\item
  The sampling distribution is set differently: based on sample VS based
  on hypothesis
\item
  Central Limit Theorem.
\end{itemize}
\end{frame}

\end{document}
